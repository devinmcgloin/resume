%%%%%%%%%%%%%%%%%%%%%%%%%%%%%%%%%%%%%%%%%
% Medium Length Professional CV
% LaTeX Template
% Version 2.0 (8/5/13)
%
% This template has been downloaded from:
% http://www.LaTeXTemplates.com
%
% Original author:
% Trey Hunner (http://www.treyhunner.com/)
%
% Important note:
% This template requires the resume.cls file to be in the same directory as the
% .tex file. The resume.cls file provides the resume style used for structuring the
% document.
%
%%%%%%%%%%%%%%%%%%%%%%%%%%%%%%%%%%%%%%%%%

%----------------------------------------------------------------------------------------
%	PACKAGES AND OTHER DOCUMENT CONFIGURATIONS
%----------------------------------------------------------------------------------------

\documentclass{resume} % Use the custom resume.cls style

\usepackage[left=0.75in,top=0.6in,right=0.75in,bottom=0.6in]{geometry} % Document margins

\name{Devin McGloin} % Your name
\address{(909)~$\cdot$~801~$\cdot$~4429 \\ devin@devinmcgloin.com} % Your phone number and email
\address{devinmcgloin.com \\ github.com/devinmcgloin} % Your phone number and email

\begin{document}

%----------------------------------------------------------------------------------------
%	EDUCATION SECTION
%----------------------------------------------------------------------------------------

\begin{rSection}{Education}

{\bf New York University} \hfill {\em May 2017} \\ 
B.S. in Computer Science \\
Global Liberal Studies \\

\end{rSection}

%----------------------------------------------------------------------------------------
%	CURRENT PROJECTS  SECTION
%----------------------------------------------------------------------------------------

\begin{rSection}{Projects}

{\bf UDB} \hfill {\em 2016} \\ 
{\em github.com/devinmcgloin/udb}\\
UDB is a universal data structure system that dynamically optimizes abstract collections. It is based on the Java Collections framework. \\

%{\bf Weekly Paper} \hfill {\em 2016}\\
%{\em devinmcgloin.com/weekly-paper}\\
%Each week I write about an influential paper in computer science, or a related field. \\ 

{\bf AIR} \hfill {\em 2015} \\ 
{\em devinmcgloin.com/AIR}\\
AIR operated under the assumption that content is key to intelligence. The central idea of the system was to represent content in a universal manner, let that content form a web of relationships, and then parse that web for information useful to NLP, and other tasks. \\

\end{rSection}

%----------------------------------------------------------------------------------------
%	WORK EXPERIENCE SECTION
%----------------------------------------------------------------------------------------

\begin{rSection}{Experience}

\begin{rSubsection}{REI}{November 2013 - Nov 2015}{Salesperson}{New York, NY}
\item Assisted customers with technical expertise regarding a vast number of merchandises and products.
\end{rSubsection}

%------------------------------------------------

\begin{rSubsection}{Classic Car Club Manhattan}{September 2013 - May 2014}{Intern}{New York, NY}
\item Managed a weekly rotating parts catalog, as well as understanding and maintaining the mechanical systems present on modern vehicles.
\item Learned the foundations of system troubleshooting, preventative maintenance as well as organizing a shop overhaul.
\end{rSubsection}

%------------------------------------------------

\begin{rSubsection}{Boy Scouts of America}{October 2009 - May 2013}{Eagle Scout}{Redlands, CA}
\item Organized and completed a service project consisting of over 600 volunteer hours and investment of \$4,000. Procured donations from the community and large donors such as Home Depot, Hines, Scotts and DIG Corp. 
\end{rSubsection}

\end{rSection}

%----------------------------------------------------------------------------------------
%	TECHNICAL STRENGTHS SECTION
%----------------------------------------------------------------------------------------

\begin{rSection}{Technical Strengths}

\begin{tabular}{ @{} >{\bfseries}l @{\hspace{6ex}} l }
Computer Languages & Java, Python, Latex, C,  \\
Protocols \& APIs & XML, JSON \\
Tools & Git, Vim, GitHub
\end{tabular}

\end{rSection}

%----------------------------------------------------------------------------------------
%	EXAMPLE SECTION
%----------------------------------------------------------------------------------------

%\begin{rSection}{Section Name}

%Section content\ldots

%\end{rSection}

%----------------------------------------------------------------------------------------

\end{document}